% FortySecondsCV LaTeX template
% Copyright © 2019 René Wirnata <rene.wirnata@pandascience.net>
% Licensed under the 3-Clause BSD License. See LICENSE file for details.
%
% Attributions
% ------------
% * fortysecondscv is based on the twentysecondcv class by Carmine Spagnuolo 
%   (cspagnuolo@unisa.it), released under the MIT license and available under
%   https://github.com/spagnuolocarmine/TwentySecondsCurriculumVitae-LaTex
% * further attributions are indicated immediately before corresponding code


%-------------------------------------------------------------------------------
%                             ADDITIONAL PACKAGES
%-------------------------------------------------------------------------------
\documentclass[
  a4paper, 
%   showframes,
%   maincolor=cvgreen,
%   sectioncolor=red,
%   subsectioncolor=orange
%   sidebarwidth=0.4\paperwidth,
%   topbottommargin=0.03\paperheight,
%   leftrightmargin=20pt
]{fortysecondscv}

% improve word spacing and hyphenation
\usepackage{microtype}
\usepackage{ragged2e}

% take care of proper font encoding
\ifxetex
	\usepackage{fontspec}
	\defaultfontfeatures{Ligatures=TeX}
% \newfontfamily\headingfont[Path = fonts/]{segoeuib.ttf} % local font
\else
	\usepackage[utf8]{inputenc}
	\usepackage[T1]{fontenc}
% \usepackage[sfdefault]{noto} % use noto google font
\fi

% enable mathematical syntax for some symbols like \varnothing
\usepackage{amssymb}

% bubble diagram configuration
\usepackage{smartdiagram}
\smartdiagramset{
  % defaut font size is \large, so adjust to harmonize with sidebar layout
  bubble center node font = \footnotesize,
  bubble node font = \footnotesize,
  % default: 4cm/2.5cm; make minimum diameter relative to sidebar size
  bubble center node size = 0.4\sidebartextwidth,
  bubble node size = 0.25\sidebartextwidth,
  distance center/other bubbles = 1.5em,
  % set center bubble color
  bubble center node color = maincolor!70,
  % define the list of colors usable in the diagram
  set color list = {maincolor!10, maincolor!40,
  maincolor!20, maincolor!60, maincolor!35},
  % sets the opacity at which the bubbles are shown
  bubble fill opacity = 0.8,
}

\usepackage{datetime}
\usepackage{lipsum}

%-------------------------------------------------------------------------------
%                            PERSONAL INFORMATION
%-------------------------------------------------------------------------------
% profile picture
\cvprofilepic{ValeLuca-29.jpg}
% your name
\cvname{Luca Di Stasio}
% job title/career
\cvjobtitle{Early Stage Researcher}
\cvcert{D-CPR Certified (FR, SE)}
\cvdlic{Driver License Cat. B (IT)}
\cvnation{Italian \& EU citizen}
% short address/location, use \newline if more than 1 line is required
\cvaddress{Stormv\"agen 299\newline SE-97634 Lule\aa, Sweden}
% phone number
\cvphone{+46 76 453 21 60}
% personal website
\cvsite{\href{https://www.lucadistasioengineering.com}{www.lucadistasioengineering.com}}
% email address
\cvmail{luca.distasio@gmail.com}
%\cvmailprof{luca.distasio@ingpec.eu}
%\linkedin{https://www.linkedin.com/in/lucadistasio/}{My Linkedin}
%\researchgate{https://www.researchgate.net/profile/Luca\_Di\_Stasio2}{My Researchgate}
%\github{https://github.com/LucaDiStasio}{My Github}

% add additional information
% \newcommand{\additional}{some more?}


%-------------------------------------------------------------------------------
%                              SIDEBAR 1st PAGE
%-------------------------------------------------------------------------------
% overwrite default icons and order of personal information
% \renewcommand{\personaltable}{%
% 	\begin{personal}[0.8em]
% 		\circleicon{\faKey}      & \cvkey  \\
% 		\circleicon{\faAt}       & \cvmail \\
% 		\circleicon{\faGlobe}    & \cvsite \\
% 		\circleicon{\faPhone}    & \cvphone \\
% 		\circleicon{\faEnvelope} & \cvaddress \\
% 		\circleicon{\faInfo}     & \cvbirthday \\
% 		% add another line
% 		\circleicon{\faQuestion} & \additional
% 	\end{personal}
% }

% add more profile sections to sidebar on first page
\addtofrontsidebar{
         	\profilesection{Contents}
		\skill{\textbf{p. 1}}{\hrulefill}
		\skill{\faBook}{Background}
		\skill{\faBook}{Higher Education Courses and Study Programmes}
		\skill{\textbf{p. 2}}{\hrulefill}
		\skill{\faBook}{Experience of Teaching and Supervision within Higher Education}
            \skill{\textbf{p. 3}}{\hrulefill}
		\skill{\faBook}{Pedagogic Activities: Description, Reflection and Development}
            \skill{\textbf{p. 4}}{\hrulefill}
		\skill{\faBook}{Development of Teaching Materials and Student Learning Resources}
		\skill{\faBook}{Experience of Leading, Administering and Developing Courses and Study Programs}
		\skill{\faBook}{Development, Depth of Study, Research and Dissemination of Knowledge}
		\skill{\faBook}{Pedagogic Activities outside the University}
}


%-------------------------------------------------------------------------------
%                              SIDEBAR 2nd PAGE
%-------------------------------------------------------------------------------
\addtobacksidebar{
\profilesection{Contents}
		\skill{\textbf{p. 1}}{\hrulefill}
		\skill{\faBook}{Background}
		\skill{\faBook}{Higher Education Courses and Study Programmes}
		\skill{\textbf{p. 2}}{\hrulefill}
		\skill{\faBook}{Experience of Teaching and Supervision within Higher Education}
            \skill{\textbf{p. 3}}{\hrulefill}
		\skill{\faBook}{Pedagogic Activities: Description, Reflection and Development}
            \skill{\textbf{p. 4}}{\hrulefill}
		\skill{\faBook}{Development of Teaching Materials and Student Learning Resources}
		\skill{\faBook}{Experience of Leading, Administering and Developing Courses and Study Programs}
		\skill{\faBook}{Development, Depth of Study, Research and Dissemination of Knowledge}
		\skill{\faBook}{Pedagogic Activities outside the University}
}

%\newdateformat{monthdayyeardate}{\monthname[\THEMONTH]~\THEDAY, \THEYEAR}%

\pagenumbering{arabic}
%-------------------------------------------------------------------------------
%                         TABLE ENTRIES RIGHT COLUMN
%-------------------------------------------------------------------------------
\begin{document}

\makefrontsidebar

\hrulefill\hspace{5pt}\textbf{\thepage}

\subsection{Background}
I am currently employed as a full-time PhD candidate in Polymeric Composite Materials at the Division of Materials Science, Department of Engineering Sciences and Mathematics, Lule\aa\ tekniska universitet (LTU) in Lule\aa, Sweden. I teach in 4 graduate-level courses offered in the subject of Polymeric Composite Materials. The courses are offered as part of the LTU-offered Master programme in Composite Materials and the international joint Master programmes in Materials Science and Engineering EEIGM/EUSMAT (European School of Materials Science and Engineering) and AMASE (Advanced Materials Science and Engineering).\\ Previously, I taught at the \'Ecole Europ\'eenne d'Ing\'enieurs en G\'enie des Mat\'eriaux (EEIGM) in Nancy, France in undergraduate- and graduate-level courses in Solid Mechanics, Viscoelasticity, Linear Elastic Fracture Mechanics, Mechanics of Composite Materials.\\I also contribute to the research activities of the Polymeric Composite Materials subject at LTU, working on integrated computational and experimental mechanics of polymers and polymer composites with a focus on fatigue, fracture and damage (see my Research Statement for more details).\\In addition, I am involved in the supervision of graduate students in the context of Master theses and project courses. I am actively involved in the continuous improvement of teaching practices in the subject of Polymeric Composite Materials by proposing new experimental activities for students (composites repair laboratory, bi-axial strain gauge measurements) as well as improving the virtual learning space of the courses offered in the subject. Furthermore, I actively contribute to the pedagogical research in Higher Education; currently I am working on a contribution (article and oral presentation) to the upcoming \textit{Development Conference for Swedish Engineering Education 2019}.

\subsection{Higher Education Courses and Study Programmes}
\textbf{Subject Related Courses}\\
As detailed in my resum\'e, I have received a BSc in Aerospace Engineering (2010) from Politecnico di Milano (Milan, Italy), a MSc in Mechanical Engineering (2012) from Drexel University (Philadelphia, USA), a MSc in Space Engineering (2013) from Politecnico di Milano (Milan, Italy), a PhD in Materials Science and Engineering (exp. Dec. 2019) from Universit\'e de Lorraine (Nancy, France) and a PhD in Polymeric Composite Materials (exp. Dec. 2019) from Lule\aa\ tekniska universitet (Lule\aa, Sweden).\\The courses I attended in these programs qualify me to teach within the specializations of Polymeric Composite Materials, Computational Mechanics, Experimental Mechanics, Computational Materials Science. I have also published peer-reviewed journal articles and conference papers and given several oral presentations in international conferences and seminars on Polymeric Composite Materials and Computational Mechanics (see my full list of publications for a more detailed account).\\[2pt]
\textbf{Pedagogic Courses}\\
I have successfully completed the 7.5 ECTS course \emph{Qualifying course for university teachers} at Lule\aa\ tekniska universitet (Lule\aa, Sweden) in February 2019.\\During my stay at the \'Ecole Europ\'eenne d'Ing\'enieurs en G\'enie des Mat\'eriaux (EEIGM) in Nancy, France, I also completed the following courses (in-presence or online) on Higher Education: Teaching in Higher Education (4 ECTS), Teaching Sustainability and Sustainable Development (2 ECTS), Oral Communication and Body Language in the Workplace (3.5 ECTS).\\Furthermore, in 2017 I completed the \textit{Software Carpentry Instructor Training Program} proposed by \textit{The Carpentries} to become a Certified Workshop Instructor. \textit{The Carpentries} is non-profit association whose aim is to teach software development and data science skills to researcher and to promote Open Science values and best practices.

\newpage
\makebacksidebar
\hrulefill\hspace{5pt}\textbf{\thepage}

\subsection{Experience of Teaching and Supervision within Higher Education}
\textbf{Teaching}\\[6pt]
\textit{Solid Mechanics, \'Ecole Europ\'eenne d'Ing\'enieurs en G\'enie des Mat\'eriaux (Nancy, France), 2017, Spring Term, Bachelor's Level ($2^{nd}$ year)}\\
I was in charge of the laboratory sessions devoted to tensile testing of aluminum, strain gauge measurements and thermomechanical measurements. I was also responsible for the examination (oral exam) and grading on this part of the course.\\[6pt]
\textit{Mechanics of Materials I, \'Ecole Europ\'eenne d'Ing\'enieurs en G\'enie des Mat\'eriaux (Nancy, France), 2017, Autumn Term, Bachelor's Level ($3^{rd}$ year)}\\
I taught tutorials to groups of 15-20 students on problems of viscoelasticity and linear elastic fracture mechanics in the form of interactive problem solving sessions. I also conducted workshops on the use of Finite Elements in materials modeling using the software Abaqus to groups of 15-20 students.\\[6pt]
\textit{Composite Materials, \'Ecole Europ\'eenne d'Ing\'enieurs en G\'enie des Mat\'eriaux (Nancy, France), 2017, Autumn Term, Master's Level ($1^{st}$ year)}\\
I conducted workshops on the use of Finite Elements in composite materials modeling using the software Abaqus to groups of 15-20 students.\\[6pt]
\textit{T7020T - Composites: Design and Numerical Methods, 7.5 ECTS, Lule\aa\ tekniska universitet (Lule\aa, Sweden), 2018, Autumn Term, Master's Level}\\
I was responsible for the laboratory sessions devoted to Mode I delamination testing (Double Cantilever Beam) of composites and calculation of Uni-Directional (UD) elastic properties from experimental data.\\[6pt]
\textit{T7005T - Aerospace Materials, 7.5 ECTS, Lule\aa\ tekniska universitet (Lule\aa, Sweden), 2018 - 2019, Spring Term, Master's Level ($1^{st}$ year)}\\
The course was divided into 3 main thematic sections: fatigue, fracture and damage in fiber-reinforced composites; joining techniques for composites; advanced metallic alloys. The first part, on fatigue, fracture and damage of composites, involved laboratory sessions of which I was in charge. In the 2019 edition of the course, I also defined the research topic of the laboratory session and designed the corresponding learning activities. I also improved the virtual learning space of the course by restructuring its content and appearance. Furthermore, I helped the design of the seminar activity in the section on advanced metallic alloys.\\[6pt]
\textit{T7012T - Composite Materials, 7.5 ECTS, Lule\aa\ tekniska universitet (Lule\aa, Sweden), 2018 - 2019, Autumn and Winter Term, Master's Level ($1^{st}$ year)}\\
I was responsible for the laboratory sessions devoted to manual manufacturing of composites, mechanical testing and calculation of Uni-Directional (UD) elastic properties from experimental data.\\[6pt]
\textit{T7011T - Mechanics of Fiber Composites, 7.5 ECTS, Lule\aa\ tekniska universitet (Lule\aa, Sweden), 2019, Winter Term, Master's Level ($1^{st}$ year)}\\
I was in charge of the laboratory sessions devoted to mechanical testing of composites and calculation of Uni-Directional (UD) elastic properties from experimental data.\\[6pt]
%%%%%%%%%%%
\textbf{Supervision}\\[6pt]
\underline{\emph{Bachelor and Master's theses and project courses}}\\[6pt]
\textit{E7009T - Degree Project, Materials Technology, 30 ECTS, Lule\aa\ tekniska universitet (Lule\aa, Sweden) and Universit\'a di Padova (Padova, Italy), 2019, Spring Term, Master's Level ($2^{nd}$ year)}\\
I co-supervised the Master's thesis work of an exchange student from Universit\'a di Padova (Padova, Italy). I met the student for 2-3 hours every week over 4 months and supported his activity by introducing him to experimental techniques, methods of design of experiments, programming languages and data analysis strategies. The student and I met with the main supervisor at Lule\aa\ tekniska universitet once every month.\\[6pt]
\textit{T7009T - Project Course, Materials Science and Engineering, 30 ECTS, Lule\aa\ tekniska universitet (Lule\aa, Sweden), 2019, Autumn Term, Master's Level ($2^{nd}$ year)}\\
I co-supervised the Project Course work of a Master level student. I formulated the research question of the project and designed the main learning activities.\\

\newpage
\makebacksidebar
\hrulefill\hspace{5pt}\textbf{\thepage}

\subsection{Pedagogic Activities: Description, Reflection and Development}
\textbf{Teaching Excellence: a Model of Target Attributes}\\[6pt]
At the beginning of my journey as a teacher in Higher Education, my original Model of Teaching Excellence was built upon my personal experience as a student and as a teacher. Being an Excellent Teacher, as I envisioned it, was thus to be the teacher I would have loved to meet when I was a student. That was the teacher I tried to be working with my first students. It was the product of my personal reflection over my experience, but no pedagogical foundation was present. A set of 8 attributes described this model of Excellent Teacher, ordered from the most to the least needed: empathic, understanding, respectful, listener, communicator, open-minded, life-long learner, expert. It was the image of a University teacher who capable of seeing each student as a responsible adult, available to help, open to different perspective and her/himself a curious learner. Expertise, meant as the expertise in one’s own field of teaching, was a less salient trait, as subject-knowledge is an ephemeral skill (today's pace of progress requires life-long learning in the scientific and technical fields) and could be developed along the way.\\[6pt]
This vision of Excellent Teacher resonated with Kierkegaard’s idea of learning facilitator [61-62]: “All genuine helping begins with humility before the person I want to help and therefore I must understand that helping is not to dominate, but to serve”.  His humanistic approach to teaching [1] was an eye-opener and represented the first critical comparison between my personal reflection on experience and the pedagogical literature. From this first encounter with Higher Education pedagogy I gained confidence and, quite frankly, a sense of relief: my reflections were in agreement with other authors, they were not only my personal views! Such realization was for me especially important, as I had been growing professionally in technical environments where teaching is usually regarded as a minor duty.\\[6pt]
Since then, my Model of Excellent Teaching and its Attributes have been evolving through a continuous comparison between experience in the classroom, personal reflection, study of pedagogical principles and experimentation of learning techniques. Attributes are still evaluated in terms of priority (5 being the highest priority), i.e. which attributes are those that must be most urgently developed. The attributes are now 7: the Excellent Teacher is a counselor (5), an active listener (5), open-minded (4), a communicator (3), a life-long learner (4), a pedagogical expert (5), a team-player (2).
The Excellent Teacher as a counselor is an attribute the foundation of which has several different aspects. Learning is affected by feelings [36]: by using Maslow’s hierarchy of needs [63] as a reference model of an individual’s motivations to act (but not as complete and perfect explanation, see my discussions in GLL4 and GLL5), it is easier to understand that unfulfillment or endangerment of lower- to mid-level needs such as safety (a basic need in Maslow’s pyramid) and esteem (a psychological need) is detrimental to learning and creative activities, which belong to the highest level (self-actualization). A student who feels that a wrong intervention in class will cause shaming by the teacher will not participate; if assessment is felt as a proof of intelligence, students would suffer a high degree of stress and likely revert to a surface approach to learning [34, 36]. Students should instead feel welcomed in their learning environment, feel free to make mistakes, free to discuss, free to express their opinions [36]. The role of the teacher is fundamental in this respect and its posture towards students influence greatly the character of the learning environment. It is in this perspective that the attributes “listener”, “emphatic”, “understanding”, “respectful” of my original MOTA acquire a more profound meaning. However, the picture is still incomplete. The learning environment should not only be welcoming and open, but it should be inclusive. The problem of inclusion can be understood by recalling the GAP model of disability [64]: a disability exists when there is a gap between an individual’s capabilities and demands from the environment. The teacher should thus be attentive to any case in which such gap between individual’s capabilities and learning activities’ requirements prevents the student from achieving the intended learning outcomes. The teacher is indeed required to act by the Swedish Discrimination Act [65] in the case the student reports a medically certified disability. However, this is a low-level behavior according to the affective taxonomic domain [30], i.e. responding or committing to ideas, values, principles and actions as mandated by the law and by the role. This is not however the level the Excellent Teacher should aim to, at least in my view: the Excellent Teacher should aim to the highest taxonomic level, i.e. internalization. The Excellent Teacher should be proactive, talk with her/his students, listen to them, understand their problems and direct them to the most suited help available. The Excellent Teacher should at the same time respect students’ privacy and not be intrusive. This behavior certainly builds upon the attributes “emphatic”, “understanding”, “respectful”, but it goes further: the Excellent Teacher should be a counselor. This motivates the substitution of “emphatic”, “understanding”, “respectful” of my original MOTA with “counselor” in my final MOTA.
In the considerations made up to now about the safe and welcoming learning environment and inclusion, listening is a skill fundamental to the Excellent Teacher. It appeared already in my original MOTA. Listening is the first fundamental step of an extremely important activity of the teacher: feedback [20, 34]. Feedback represents the final of the Four Questions [13-18] and the chance for the teacher to verify the alignment [19, 20, 50] of the learning activity performed with the intended learning outcomes. For the student, feedback shows whether the effort spent has produced the expected learning, and how it should or could be improved. For the feedback to be effective, the teacher should listen and should understand what the student is communicating: understanding cannot be shallow, i.e. only of the literal meaning, but deep, i.e. of the intended content. This is what is called “active listening” in the counseling community [66, 67]. The Excellent Teacher should not only be a listener (as in my original MOTA), but more specifically an active listener (as in my final version of the MOTA). On the other hand, to receive feedback and use it for improvement, the Excellent Teacher must be open to critiques on her/his own work: the Excellent Teacher should thus be “open-minded”. Communication appears to be at heart of the teaching activity: creation of a safe and welcoming environment, inclusion, feedback, all require good communication skills. Thus, the Excellent Teacher needs to be a good communicator. This attribute is present both in my initial and my final MOTA, although in the latter it is meaningful in relation to the safe and welcoming environment, inclusion and feedback.
Motivation is key to students’ engagement and focus [36, 42-48] and intrinsic motivated students tend to prefer deep approaches to learning [36, 47, 48]. One of the best and probably simplest way for the teacher to relate to students’ intrinsic motivation is to have and show a strong interest and passion for the subject taught and to present its links with the future students’ profession and the problems of the world at large. This is achieved if the teacher is a life-long learner. It means that the teacher has a strong desire to learn and improve and the teaching assignment is felt by the teacher first and foremost as a learning moment. Therefore, the Excellent Teacher needs to be a life-long learner and this attribute is still present in my final MOTA.
The experience of the “Teach the Teachers” session has showed me how important is the design of learning activities following the principles of constructive alignment [23-27] and using tools such as the taxonomies [29-31], Kolb’s cycle [40] and the ICAP framework [49]. Pedagogical expertise is a fundamental skill of the teacher, who should be able to correctly formulate learning outcomes and design learning activities aligned with these outcomes. It thus substitutes the subject-specific expertise of my original MOTA: in the final version of the MOTA, the Excellent Teacher should be a pedagogical expert. The group work on the “Teach the Teachers” workshop has helped to understand an additional trait of the Excellent Teacher, which was completely absent in my original MOTA: the Excellent Teacher as a team-player. Preparation of a learning session in team certainly takes more time than working alone, but the quality is higher. The learning session is better designed, and the design process becomes a possibility of growth for teachers through the constant relation with peers, and the discussion of pedagogical stand-points and design issues.

\newpage
\makebacksidebar
\hrulefill\hspace{5pt}\textbf{\thepage}

\subsection{Development of Teaching Materials and Student Learning Resources}
\textit{T7005T - Aerospace Materials, 7.5 ECTS, Lule\aa\ tekniska universitet (Lule\aa, Sweden), 2018 - 2019, Spring Term, Master's Level ($1^{st}$ year)}\\
I participated in the preparation of the hand-out, in the form of a presentation, describing the tasks of the group-based project. I worked on improving the virtual learning space of the course by organizing its content and structuring its appearance.

\subsection{Experience of Leading, Administering and Developing Courses and Study Programs}
As part of the coursework for the 7.5 ECTS course \emph{Qualifying course for university teachers} at Lule\aa\ tekniska universitet (Lule\aa, Sweden), I re-designed the 7.5 ECTS course \textit{T7005T - Aerospace Materials} (see attachment for more details). To this end, I proposed a new format of flipped-classroom learning for the theoretical part of the course and a more articulated management of the group-based project. The core idea of this new design is to bring passive-active (with respect to the ICAP framework), lower taxonomy level activities outside of the classroom and focus the in-class work on constructive and interactive behaviors aimed to the construction of new knowledge. As this course is propedeutic to independent research projects in the last year of studies, the new version of the course would provide a fail-safe training ground similar to the daily activity of the researcher and the engineer, who develop projects with uncertain initial boundaries and need to interact with co-workers, team leaders, customers and stakeholders.

\subsection{Development, Depth of Study, Research and Dissemination of Knowledge}
I am currently preparing a contribution in the form of conference article and oral presentation, which has been accepted for the upcoming \textit{Development Conference for Swedish Engineering Education 2019} to be held at Lule\aa\ tekniska universitet (Lule\aa, Sweden) in November 2019. In this contribution I present, in the form of a work in progress, a reflection on the relation between the signature pedagogy of mechanics of materials and its threshold concepts, and I further propose a revised flipped-classroom format for lectures on mechanics of materials based on the signature pedagogy. In the long-term, the goal is to be experiment this revised flipped-classroom format in a course on mechanics of materials and verify its effectiveness through students' feedback and review from peers. The results would then be amenable for publication in a peer-reviewed international journal to get feedback from the wider community of practitioners in Higher Education.

\subsection{Pedagogic Activities outside the University}
I am an active member of the non-profit organization \emph{The Carpentries}, with which I collaborate on a volunteer basis. The main objective of the organization is to teach foundational skills in software development and data analysis to researchers and academics at all levels (from students to professors) and to promote Open Science values and best practices. To achieve this goal, it offers free online tutorials on different technical topics and organizes, upon request from local institutions, in-person workshops. To this end, the organization trains its volunteer members on active learning strategies. To distinguish members who have acquired such skills, \emph{The Carpentries} proposes a specific training path which, upon successful completion, provides the status of \emph{Certified Instructor} and the possibility to teach in workshops. I completed the instructor training in 2017 and I am, since then, a \emph{Certified Instructor}. In October 2018, I participated as main instructor in a Carpentry workshop offered by the High Performance Computing Center North (HPC2N) at Ume\aa\ University (Ume\aa, Sweden), where I taught the use of bash, bash scripting, Git and Github. In addition, I contributed to the development of the workshop lessons material, which is freely available on Github. I contributed also to a Task Force to develop recommendations on handling incidents outside the Code of Conduct of the \emph{The Carpentries}.

\end{document} 

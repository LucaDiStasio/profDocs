% FortySecondsCV LaTeX template
% Copyright © 2019 René Wirnata <rene.wirnata@pandascience.net>
% Licensed under the 3-Clause BSD License. See LICENSE file for details.
%
% Attributions
% ------------
% * fortysecondscv is based on the twentysecondcv class by Carmine Spagnuolo 
%   (cspagnuolo@unisa.it), released under the MIT license and available under
%   https://github.com/spagnuolocarmine/TwentySecondsCurriculumVitae-LaTex
% * further attributions are indicated immediately before corresponding code


%-------------------------------------------------------------------------------
%                             ADDITIONAL PACKAGES
%-------------------------------------------------------------------------------
\documentclass[
  a4paper, 
%   showframes,
%   maincolor=cvgreen,
%   sectioncolor=red,
%   subsectioncolor=orange
%   sidebarwidth=0.4\paperwidth,
%   topbottommargin=0.03\paperheight,
%   leftrightmargin=20pt
]{fortysecondscv}

% improve word spacing and hyphenation
\usepackage{microtype}
\usepackage{ragged2e}

% take care of proper font encoding
\ifxetex
	\usepackage{fontspec}
	\defaultfontfeatures{Ligatures=TeX}
% \newfontfamily\headingfont[Path = fonts/]{segoeuib.ttf} % local font
\else
	\usepackage[utf8]{inputenc}
	\usepackage[T1]{fontenc}
% \usepackage[sfdefault]{noto} % use noto google font
\fi

% enable mathematical syntax for some symbols like \varnothing
\usepackage{amssymb}

% bubble diagram configuration
\usepackage{smartdiagram}
\smartdiagramset{
  % defaut font size is \large, so adjust to harmonize with sidebar layout
  bubble center node font = \footnotesize,
  bubble node font = \footnotesize,
  % default: 4cm/2.5cm; make minimum diameter relative to sidebar size
  bubble center node size = 0.4\sidebartextwidth,
  bubble node size = 0.25\sidebartextwidth,
  distance center/other bubbles = 1.5em,
  % set center bubble color
  bubble center node color = maincolor!70,
  % define the list of colors usable in the diagram
  set color list = {maincolor!10, maincolor!40,
  maincolor!20, maincolor!60, maincolor!35},
  % sets the opacity at which the bubbles are shown
  bubble fill opacity = 0.8,
}

\usepackage{datetime}
\usepackage{lipsum}

%-------------------------------------------------------------------------------
%                            PERSONAL INFORMATION
%-------------------------------------------------------------------------------
% profile picture
\cvprofilepic{ValeLuca-29.jpg}
% your name
\cvname{Luca Di Stasio}
% job title/career
\cvjobtitle{Early Stage Researcher}
\cvcert{D-CPR Certified (FR, SE)}
\cvdlic{Driver License Cat. B (IT)}
\cvnation{Italian \& EU citizen}
% short address/location, use \newline if more than 1 line is required
\cvaddress{Stormv\"agen 299\newline SE-97634 Lule\aa, Sweden}
% phone number
\cvphone{+46 76 453 21 60}
% personal website
\cvsite{\href{https://www.lucadistasioengineering.com}{www.lucadistasioengineering.com}}
% email address
\cvmail{luca.distasio@gmail.com}
%\cvmailprof{luca.distasio@ingpec.eu}
%\linkedin{https://www.linkedin.com/in/lucadistasio/}{My Linkedin}
%\researchgate{https://www.researchgate.net/profile/Luca\_Di\_Stasio2}{My Researchgate}
%\github{https://github.com/LucaDiStasio}{My Github}

% add additional information
% \newcommand{\additional}{some more?}


%-------------------------------------------------------------------------------
%                              SIDEBAR 1st PAGE
%-------------------------------------------------------------------------------
% overwrite default icons and order of personal information
% \renewcommand{\personaltable}{%
% 	\begin{personal}[0.8em]
% 		\circleicon{\faKey}      & \cvkey  \\
% 		\circleicon{\faAt}       & \cvmail \\
% 		\circleicon{\faGlobe}    & \cvsite \\
% 		\circleicon{\faPhone}    & \cvphone \\
% 		\circleicon{\faEnvelope} & \cvaddress \\
% 		\circleicon{\faInfo}     & \cvbirthday \\
% 		% add another line
% 		\circleicon{\faQuestion} & \additional
% 	\end{personal}
% }

% add more profile sections to sidebar on first page
\addtofrontsidebar{
         	
		
}


%-------------------------------------------------------------------------------
%                              SIDEBAR 2nd PAGE
%-------------------------------------------------------------------------------
\addtobacksidebar{

}

%\newdateformat{monthdayyeardate}{\monthname[\THEMONTH]~\THEDAY, \THEYEAR}%


%-------------------------------------------------------------------------------
%                         TABLE ENTRIES RIGHT COLUMN
%-------------------------------------------------------------------------------
\begin{document}

\makefrontsidebar

\subsection{Background}
I am currently employed as a full-time PhD candidate in Polymeric Composite Materials at the Division of Materials Science, Department of Engineering Sciences and Mathematics, Lule\aa\ tekniska universitet (LTU) in Lule\aa, Sweden. I currently teach in 4 graduate-level courses offered in the subject of Polymeric Composite Materials. The courses are offered as part of the LTU-offered Master programme in Composite Materials and the international joint Master programmes in Materials Science and Engineering EEIGM/EUSMAT (European School of Materials Science and Engineering) and AMASE (Advanced Materials Science and Engineering). Previously, I taught at the \'Ecole Europ\'eenne d'Ing\'enieurs en G\'enie des Mat\'eriaux (EEIGM) in Nancy, France in undergraduate- and graduate-level courses in Solid Mechanics, Viscoelasticity, Linear Elastic Fracture Mechanics, Mechanics of Composite Materials. I also contribute to the research activities of the Polymeric Composite Materials subject at LTU, working on integrated computational and experimental mechanics of polymers and polymer composites with a focus on fatigue, fracture and damage (see my Research Statement for more details). In addition, I am involved in the supervision of graduate students in the context of Master theses and project courses. I am actively involved in the continuous improvement of teaching practices in the subject of Polymeric Composite Materials by proposing new experimental activities for students (composites repair laboratory, bi-axial strain gauge measurements) as well as improving the virtual learning space of the courses offered in the subject. Furthermore, I actively contribute to the pedagogical research in Higher Education; currently I am working on a contribution (article and oral presentation) to the upcoming \textit{Development Conference for Swedish Engineering Education 2019}.

\subsection{Higher Education Courses and Study Programmes}
\textbf{Subject Related Courses}\\
As detailed in my resum\'e, I have received a BSc in Aerospace Engineering (2010) from Politecnico di Milano (Milan, Italy), a MSc in Mechanical Engineering (2012) from Drexel University (Philadelphia, USA), a MSc in Space

\subsection{Experience of Teaching and Supervision within Higher Education}
In the short- to medium-term perspective, I am currently laying the groundwork for several future works: derivation of the vectorial VCCT from Eshelby's elastic energy-momentum tensor and proposal of a new mode-partitioning strategy based on eigenvalue analysis; investigation of fiber/matrix debonding with concurrent non-linear (viscoelastic, viscoplastic) behavior of the surrounding matrix; 3D modeling of fiber/matrix debonding; 3D imaging of fiber/matrix debonding using in-situ micro-tomography; development of an image analysis algorithm for automated stress-free temperature identification through curvature measurements of asymmetric laminates and its implementation with a temperature feedback loop on low-cost hardware (e.g. android handset, Arduino, Raspberry Pi); creation and real-time update of specimens' digital twins through low-cost hardware (Kinect for Xbox); application of Bayesian inference to the prediction of elastic, viscoelastic, viscoplastic, failure and fracture toughness properties.\\
In the long term, I envision the development of the distributed, de-centralized, remotely-controlled, integrated laboratory for composite science and engineering: a set of fully automated laboratories and high-performance computing clusters connected together by a decentralized network (peer-to-peer) and accessible through an online platform, to allow collaborative projects on integrated computational-experimental analysis and design of materials between parties located around the globe.

\end{document} 

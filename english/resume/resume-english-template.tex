%%%%%%%%%%%%%%%%%%%%%%%%%%%%%%%%
%%%%%%%%%%%%%%%%%%%%%%%%%%%%%%%%

%                                  CURRICULUM VITAE                                
%                                    ENGLISH VERSION                                  

%%%%%%%%%%%%%%%%%%%%%%%%%%%%%%%%

% Copyright (c) 2015 - 2019 Luca Di Stasio
% Author: Luca Di Stasio <luca.distasio@gmail.com>
%                                  <luca.distasio@ingpec.eu>
%
% This program is free software: you can redistribute it and/or modify
% it under the terms of the GNU General Public License as published by
% the Free Software Foundation, either version 3 of the License, or
% (at your option) any later version.
%
% This program is distributed in the hope that it will be useful,
% but WITHOUT ANY WARRANTY; without even the implied warranty of
% MERCHANTABILITY or FITNESS FOR A PARTICULAR PURPOSE.  See the
% GNU General Public License for more details.
%
% You should have received a copy of the GNU General Public License
% along with this program.  If not, see <http://www.gnu.org/licenses/>.

%%%%%%%%%%%%%%%%%%%%%%%%%%%%%%%%
%%%%%%%%%%%%%%%%%%%%%%%%%%%%%%%%

%----------------------------------------------------------------------------------------
%	PACKAGES AND OTHER DOCUMENT CONFIGURATIONS
%----------------------------------------------------------------------------------------

\documentclass[letterpaper]{twentysecondcv} % a4paper for A4

%----------------------------------------------------------------------------------------
%	 PERSONAL INFORMATION
%----------------------------------------------------------------------------------------

% If you don't need one or more of the below, just remove the content leaving the command, e.g. \cvnumberphone{}

\profilepic{Luca_Di_Stasio_photo_reduced.jpeg} % Profile picture

\cvname{Luca Di Stasio} % Your name
\cvjobtitle{Early Stage Researcher} % Job title/career

\cvdate{April 19$^{th}$, 1988} % Date of birth
\cvaddress{Stormv\"agen 299\newline 97643 Lule\aa, Sweden} % Short address/location, use \newline if more than 1 line is required
\cvnumberphone{+46 76 453 21 60} % Phone number
\cvsite{\href{https://www.linkedin.com/in/lucadistasio}{\textit{Linkedin profile}}} % Personal website
\cvmail{luca.distasio@gmail.com} % Email address

%----------------------------------------------------------------------------------------

\begin{document}

%----------------------------------------------------------------------------------------
%	 ABOUT ME
%----------------------------------------------------------------------------------------

\aboutme{Alice is a sensible prepubescent girl from a wealthy English family who finds herself in a strange world ruled by imagination and fantasy. Alice feels comfortable with her identity and has a strong sense that her environment is comprised of clear, logical, and consistent rules and features. Alice's familiarity with the world has led one critic to describe her as a "disembodied intellect". Alice displays great curiosity and attempts to fit her diverse experiences into a clear understanding of the world.} % To have no About Me section, just remove all the text and leave \aboutme{}

%----------------------------------------------------------------------------------------
%	 SKILLS
%----------------------------------------------------------------------------------------

% Skill bar section, each skill must have a value between 0 an 6 (float)
\skills{{pursuer of rabbits/5.8},{good manners/4},{outgoing/4.3},{polite/4},{Java/0.01}}

%------------------------------------------------

% Skill text section, each skill must have a value between 0 an 6
\skillstext{{lovely/4},{narcissistic/3}}

%----------------------------------------------------------------------------------------

\makeprofile % Print the sidebar

%----------------------------------------------------------------------------------------
%	 INTERESTS
%----------------------------------------------------------------------------------------

\section{Interests}

The heroine and the dreamer of Wonderland; Alice is the principal character.

%----------------------------------------------------------------------------------------
%	 EDUCATION
%----------------------------------------------------------------------------------------

\section{Education}

\begin{twenty} % Environment for a list with descriptions
	\twentyitem{since 1865}{Ph.D. {\normalfont candidate in Computer Science}}{Wonderland}{\emph{A Quantified Theory of Social Cohesion.}}
	\twentyitem{1863-1865}{M.Sc. magna cum laude}{Wonderland}{Majoring in Computer Science}
	\twentyitem{1861-1863}{B.Sc. magna cum laude}{Wonderland}{Majoring in Computer Science}
	\twentyitem{1856-1861}{High school}{Wonderland}{Specializing in mathematics and physics.}
	%\twentyitem{<dates>}{<title>}{<location>}{<description>}
\end{twenty}

\begin{twenty}
\twentyitem{1234}{lalal}{lalala}{lalalalalalla}
\end{twenty}

%----------------------------------------------------------------------------------------
%	 PUBLICATIONS
%----------------------------------------------------------------------------------------

\section{Publications}

\begin{twentyshort} % Environment for a short list with no descriptions
	\twentyitemshort{1865}{Chapter One, Down the Rabbit Hole.}
	\twentyitemshort{1865}{Chapter Two, The Pool of Tears.}
	\twentyitemshort{1865}{Chapter Three,  The Caucus Race and a Long Tale.}
	\twentyitemshort{1865}{Chapter Four,  The Rabbit Sends a Little Bill.}
	\twentyitemshort{1865}{Chapter Five,  Advice from a Caterpillar.}
	%\twentyitemshort{<dates>}{<title/description>}
\end{twentyshort}

%----------------------------------------------------------------------------------------
%	 AWARDS
%----------------------------------------------------------------------------------------

\section{Awards}

\begin{twentyshort} % Environment for a short list with no descriptions
	\twentyitemshort{1987}{All-Time Best Fantasy Novel.}
	\twentyitemshort{1998}{All-Time Best Fantasy Novel before 1990.}
	%\twentyitemshort{<dates>}{<title/description>}
\end{twentyshort}

%----------------------------------------------------------------------------------------
%	 EXPERIENCE
%----------------------------------------------------------------------------------------

\section{Experience}

\begin{twenty} % Environment for a list with descriptions
	\twentyitem{1900}{Alice in Wonderland-The Circra (1900's) Silent Film.}{Film}{The first Alice on film was over a hundred years ago.}
	\twentyitem{1933}{Alice in Wonderland 1933 version.}{Film}{This film stars Ethel griffies and Charlotte Henry. It was a box office flop when it was released.}
	\twentyitem{1951}{Disney Film.}{Film}{Walt Disney brings Lewis Carroll's fantasy story to life in this well done animated classic. Even though many elements from the book were dropped, such as the duchess with the baby pig and mock turtle, this version is without a doubt the most famous Alice adaption made.}
	%\twentyitem{<dates>}{<title>}{<location>}{<description>}
\end{twenty}

%----------------------------------------------------------------------------------------
%	 OTHER INFORMATION
%----------------------------------------------------------------------------------------

\section{Other information}

\subsection{Review}

Alice approaches Wonderland as an anthropologist, but maintains a strong sense of noblesse oblige that comes with her class status. She has confidence in her social position, education, and the Victorian virtue of good manners. Alice has a feeling of entitlement, particularly when comparing herself to Mabel, whom she declares has a ``poky little house," and no toys. Additionally, she flaunts her limited information base with anyone who will listen and becomes increasingly obsessed with the importance of good manners as she deals with the rude creatures of Wonderland. Alice maintains a superior attitude and behaves with solicitous indulgence toward those she believes are less privileged.

%----------------------------------------------------------------------------------------
%	 SECOND PAGE EXAMPLE
%----------------------------------------------------------------------------------------

%\newpage % Start a new page

%\makeprofile % Print the sidebar

%\section{Other information}

%\subsection{Review}

%Alice approaches Wonderland as an anthropologist, but maintains a strong sense of noblesse oblige that comes with her class status. She has confidence in her social position, education, and the Victorian virtue of good manners. Alice has a feeling of entitlement, particularly when comparing herself to Mabel, whom she declares has a ``poky little house," and no toys. Additionally, she flaunts her limited information base with anyone who will listen and becomes increasingly obsessed with the importance of good manners as she deals with the rude creatures of Wonderland. Alice maintains a superior attitude and behaves with solicitous indulgence toward those she believes are less privileged.

%\section{Other information}

%\subsection{Review}

%Alice approaches Wonderland as an anthropologist, but maintains a strong sense of noblesse oblige that comes with her class status. She has confidence in her social position, education, and the Victorian virtue of good manners. Alice has a feeling of entitlement, particularly when comparing herself to Mabel, whom she declares has a ``poky little house," and no toys. Additionally, she flaunts her limited information base with anyone who will listen and becomes increasingly obsessed with the importance of good manners as she deals with the rude creatures of Wonderland. Alice maintains a superior attitude and behaves with solicitous indulgence toward those she believes are less privileged.

%----------------------------------------------------------------------------------------

\end{document}
%%%%%%%%%%%%%%%%%%%%%%%%%%%%%%%%
%%%%%%%%%%%%%%%%%%%%%%%%%%%%%%%%

%                                  CURRICULUM VITAE                                
%                                    ENGLISH VERSION                                  

%%%%%%%%%%%%%%%%%%%%%%%%%%%%%%%%

% Copyright (c) 2015 - 2018 Luca Di Stasio
% Author: Luca Di Stasio <luca.distasio@gmail.com>
%                                     <luca.distasio@ingpec.eu>
%
% This program is free software: you can redistribute it and/or modify
% it under the terms of the GNU General Public License as published by
% the Free Software Foundation, either version 3 of the License, or
% (at your option) any later version.
%
% This program is distributed in the hope that it will be useful,
% but WITHOUT ANY WARRANTY; without even the implied warranty of
% MERCHANTABILITY or FITNESS FOR A PARTICULAR PURPOSE.  See the
% GNU General Public License for more details.
%
% You should have received a copy of the GNU General Public License
% along with this program.  If not, see <http://www.gnu.org/licenses/>.

%%%%%%%%%%%%%%%%%%%%%%%%%%%%%%%%
%%%%%%%%%%%%%%%%%%%%%%%%%%%%%%%%

%%%%%%%%%%%%%%%%%%%%%%%%%%%%%
%                                                                                                    %
%                                   Important note:                                       %
%                                                                                                    %
%          This template requires the moderncv.cls and .sty            %
%          files to be in the same directory as this .tex file.              %
%          These files provide the resume style and themes             %
%                     used for structuring the document.                       %
%                                                                                                    %
%%%%%%%%%%%%%%%%%%%%%%%%%%%%%

%%%%%%%%%%%%%%%%%%%%%%%%%%%%%
%%%%%%%%%%%%%%%%%%%%%%%%%%%%%
%%---------------------------------------------------------------------%%
%%------------------------ PREAMBLE -------------------------------%%
%%---------------------------------------------------------------------%%
%%%%%%%%%%%%%%%%%%%%%%%%%%%%%
%%%%%%%%%%%%%%%%%%%%%%%%%%%%%


%        BASIC DOCUMENT DECLARATIONS       

\documentclass[11pt,a4paper,sans]{moderncv} % Font sizes: 10, 11, or 12; paper sizes: a4paper, letterpaper, a5paper, legalpaper, executivepaper or landscape; font families: sans or roman

\moderncvstyle{casual} % CV theme - options include: 'casual' (default), 'classic', 'oldstyle' and 'banking'
\moderncvcolor{blue} % CV color - options include: 'blue' (default), 'orange', 'green', 'red', 'purple', 'grey' and 'black'

%----------------------------------------------%
%                 PACCHETTI                    %
%----------------------------------------------%

\usepackage{lipsum} % Used for inserting dummy 'Lorem ipsum' text into the template

\usepackage[scale=0.75]{geometry} % Reduce document margins
%\setlength{\hintscolumnwidth}{3cm} % Uncomment to change the width of the dates column
%\setlength{\makecvtitlenamewidth}{10cm} % For the 'classic' style, uncomment to adjust the width of the space allocated to your name

\usepackage{ wasysym }

%----------------------------------------------%
%       NOME ED INFORMAZIONI DI CONTATTO       %
%----------------------------------------------%

\firstname{Luca} % Your first name
\familyname{Di Stasio} % Your last name

% All information in this block is optional, comment out any lines you don't need
\title{Curriculum Vitae}
\photo[70pt][0.4pt]{pictures/Luca_Di_Stasio_photo_reduced} % The first bracket is the picture height, the second is the thickness of the frame around the picture (0pt for no frame)
%\quote{"A witty and playful quotation" - John Smith}

%%%%%%%%%%%%%%%%%%%%%%%%%%%%%%%%%%%%%%%%%%%%%%%%%%%%%%%%%%%%%%%%%%%%%%%%%%%%%%%%%%%%%%%%%%%%%%%%%%%%%%
%%%%%%%%%%%%%%%%%%%%%%%%%%%%%%%%%%%%%%%%%%%%%%%%%%%%%%%%%%%%%%%%%%%%%%%%%%%%%%%%%%%%%%%%%%%%%%%%%%%%%%
%%--------------------------------------------------------------------------------------------------%%
%%--------------------------------- INIZIO DEL DOCUMENTO -------------------------------------------%%
%%--------------------------------------------------------------------------------------------------%%
%%%%%%%%%%%%%%%%%%%%%%%%%%%%%%%%%%%%%%%%%%%%%%%%%%%%%%%%%%%%%%%%%%%%%%%%%%%%%%%%%%%%%%%%%%%%%%%%%%%%%%
%%%%%%%%%%%%%%%%%%%%%%%%%%%%%%%%%%%%%%%%%%%%%%%%%%%%%%%%%%%%%%%%%%%%%%%%%%%%%%%%%%%%%%%%%%%%%%%%%%%%%%

\begin{document}

\makecvtitle % Print the CV title

%----------------------------------------------%
%            INFORMAZIONI PERSONALI            %
%----------------------------------------------%

\vspace*{0.5cm}

\section{Informazioni di contatto professionali}

\cvitem{\textit{Indirizzo}}{Ecole Europ\'eenne d'Ing\'enieurs en G\'enie des Mat\'eriaux (EEIGM)}
\cvitem{\textit{}}{Universit\'e de Lorraine}
\cvitem{}{6 Rue Bastien-Lepage, CS 10630, F-54010, Nancy - Francia}
%\cvitem{\textit{Mobile}}{3921673742}
%\cvitem{\textit{Telefono}}{(FR) +33 03 83 36 75 05}
%\cvitem{\textit{Fax}}
\cvitem{\textit{Email}}{\href{mailto:luca.distasio@ingpec.eu}{luca.distasio@ingpec.eu}}
\cvitem{\textit{}}{\href{mailto:luca.di-stasio@univ-lorraine.fr}{luca.di-stasio@univ-lorraine.fr}}
%\cvitem{\textit{Webpage}}{\href{http://www.ifb.ethz.ch/comphys/people/lucad}{www.ifb.ethz.ch/comphys/people/lucad}}
%\cvitem{\textit{Extra}}

\section{Informazioni di contatto personali}

\cvitem{\textit{Indirizzo (domicilio)}}{40 Rue de Laxou, F-54000, Nancy - Francia}
\cvitem{\textit{Indirizzo (residenza)}}{via Verdi 64, 20063 Cernusco sul Naviglio (MI) - Italia}
\cvitem{\textit{Cellulare}}{(FR) +33 06 95 87 76 17}
\cvitem{\textit{}}{(IT) \ +39 392 16 73 742}
%\cvitem{\textit{Phone}}
%\cvitem{\textit{Fax}}
\cvitem{\textit{Email}}{\href{mailto:luca.distasio@gmail.com}{luca.distasio@gmail.com}}
\cvitem{\textit{Webpage}}{\href{http://www.lucadistasioengineering.com/}{\textit{La mia pagina personale}}}
%\cvitem{\textit{}}{\href{http://www.lucadistasioengineering.com}{www.lucadistasioengineering.com}}
%\cvitem{\textit{}}{\href{https://www.linkedin.com/profile/view?id=127761965&locale=en_US&trk=profile_view_lang_sel_click}{Profilo LinkedIn}}
%\cvitem{\textit{Extra}}

\section{Informazioni personali addizionali}

\cvitem{\textit{Nazionalit\`a}}{Italiana}
\cvitem{\textit{Luogo di nascita}}{Cernusco s/N, Milano, Italia}
\cvitem{\textit{Data di nascita}}{19 aprile 1988}
\cvitem{\textit{Stato civile}}{Fidanzato}
%\cvitem{\textit{Permessi di soggiorno}}{Permesso di residenza B svizzero (permesso di residenza quinquennale)}
\cvitem{\textit{Patenti di guida}}{Patente di guida italiana, cat. B}

\newpage

%----------------------------------------------%
%          ESPERIENZA PROFESSIONALE            %
%----------------------------------------------%

\section{Esperienza professionale}

%------------------------------------------------

\subsection{Attivit\`a ed interessi di ricerca}

\cventry{Ott. 2015 - Presente}{Assegnista di ricerca pre-dottorale}{Universit\'e de Lorraine}{}{Nancy, Francia}{\emph{Mechanics of Materials Group, Department of Metallurgical and Materials Science and Engineering (SI2M), Institute Jean Lamour, Ecole Europ\'eenne d'Ing\'enieurs en G\'enie des Mat\'eriaux (EEIGM), Universit\'e de Lorraine}}
\cvitem{}{\small \textit{Progetto:} Meccanica di lamine composite ultra-fini per applicazioni aerospaziali.}
\cvitem{}{\small \textit{Lingua di lavoro:} francese, inglese.}

\cventry{2013 - 2015}{Assistente di ricerca}{ETH Z\"urich}{}{Zurigo, Svizzera}{\emph{Computational Physics for Engineering Materials Group, Institute for Building Materials, Department of Civil, Environmental and Geomatic Engineering, ETH Z\"urich}}
\cvitem{}{\small \textit{Progetto:} Modellazione multi-scala del legno.}
\cvitem{}{\small \textit{Progetto:} Interazione fluido-struttura su superfici deformabili.}
\cvitem{}{\small \textit{Lingua di lavoro:} tedesco, inglese.}

\cventry{2012 - 2013}{Assistente di ricerca}{IMDEA Materials Institute}{}{Getafe, Madrid, Spagna}{\emph{Structural Composites Group, IMDEA Materials Institute}}
\cvitem{}{\small \textit{Progetto:} Effetti della velocit\`a di carico sulla propagazione della delaminazione (modo I, modo II e modo misto I-II) in laminati polimerici compositi con fibra di carbonio.}
\cvitem{}{\small \textit{Lingua di lavoro:} spagnolo, inglese.}

\cventry{Gen. - Giu. 2012}{Assistente di ricerca}{Drexel University}{}{Philadelphia, USA}{\emph{Mesoscale Materials Laboratory, Department of Materials Science \& Engineering, Institute for Energy \& the Environment, Drexel University}}
\cvitem{}{\small \textit{Progetto:} Progetto di nano-resonatori per il rilevamento ambientale di campi elettro-magnetici.}
\cvitem{}{\small \textit{Lingua di lavoro:} inglese.}

\cventry{Gen. - Giu. 2012}{Assistente di ricerca}{Drexel University}{}{Philadelphia, USA}{\emph{BioMechanics Laboratory, Department of Mechanical Engineering \& Mechanics, Drexel University}}
\cvitem{}{\small \textit{Progetto:} Propriet\`a morfologiche dell'astragalo e loro relazione con la cinematica della caviglia.}
\cvitem{}{\small \textit{Lingua di lavoro:} inglese.}

%------------------------------------------------

\subsection{Attivit\`a ed interessi di insegnamento}

\cventry{2014 - 2015}{Tutor}{The Learning Center Z\"urich}{Zurigo, Svizzera}{}{Ho lavorato in qualit\`a di insegnante per attivit\`a extra-curricolari di aiuto allo studio per studenti di liceo ed universit\'a.}
\cvitem{}{\small \textit{Discipline:} Matematica, Fisica, Statistica, Econometria, programmazione Java.}
\cvitem{}{\small \textit{Lingua di lavoro:} tedesco, inglese.}

\cventry{Mar. - Lug. 2011}{Tutor}{Politecnico di Milano}{Milano, Italia}{}{Ho lavorato in qualit\`a di tutor di uno studente di Laurea triennale con disabilit\'a.}
\cvitem{}{\small \textit{Discipline:} Analisi matematica.}
\cvitem{}{\small \textit{Lingua di lavoro:} italiano.}

\cventry{2009 - 2011}{Tutor}{Liceo Scientifico Istituto Sacro Cuore}{Milano, Italia}{}{Ho lavorato in qualit\`a di insegnante per attivit\`a extra-curricolari di aiuto allo studio per studenti di liceo.}
\cvitem{}{\small \textit{Discipline:} Matematica, Fisica, Biologia, Lingua e letteratura italiane, Lingua e letteratura latine, storia.}
\cvitem{}{\small \textit{Lingua di lavoro:} italiano.}

%------------------------------------------------

\subsection{Attivit\`a istituzionali}

\cventry{2008 - 2011}{Rappresentante degli studenti}{Politecnico di Milano}{Milano, Italia}{}{Rappresentante degli studenti nel Consiglio di Corso di Studi del Dipartimento di Ingegneria Aerospaziale del Politecnico di Milano.}
\cvitem{}{\small \textit{Lingua di lavoro:} italiano.}

\cventry{2008 - 2011}{Rappresentante degli studenti}{Politecnico di Milano}{Milano, Italia}{}{Rappresentante degli studenti nel Consiglio della Facolt\`a di Ingegneria Industriale del Politecnico di Milano.}
\cvitem{}{\small \textit{Lingua di lavoro:} italiano.}

%------------------------------------------------

\subsection{Attivit\`a artigianali e lavoro manuale}

\cventry{2007 - 2009}{Conducente}{ADT}{Cernusco s/N, Milano, Italia}{}{Ho lavorato in qualit\`a di conducente presso un'azienda di trasporti a conduzione familiare.}
\cvitem{}{\small \textit{Lingua di lavoro:} italiano.}

\cventry{2005 - 2007}{Operaio}{Delta I Srl.}{Cernusco s/N, Milano, Italia}{}{Ho lavorato in qualit\`a di operaio presso un'azienda di sistemi elettro-meccanici a conduzione familiare, dove ho imparato ad operare con diverse macchine e processi di produzione.}
\cvitem{}{\small \textit{Lingua di lavoro:} italiano.}

%------------------------------------------------

\subsection{Attivit\`a di volontariato}

\cventry{2008 - 2011}{Tecnico audio}{CLU (associazione cattolica)}{Milano, Italia}{}{Come membro di un'associazione cattolica non a scopo di lucro, ho lavorato come volontario in qualit\`a di tecnico audio per la preparazione di eventi ed incontri.}
\cvitem{}{\small \textit{Lingua di lavoro:} italiano.}

\cventry{2008 - 2011}{Segretario}{CLU (associazione cattolica)}{Milano, Italia}{}{Come membro di un'associazione cattolica non a scopo di lucro, ho lavorato come volontario in qualit\`a di segretario, con compiti di amministrazione, contabilit\`a, pianificazione e gestione di eventi.}
\cvitem{}{\small \textit{Lingua di lavoro:} italiano.}

\cventry{2005 - 2007}{Catechista}{Parrocchia Madonna del Divin Pianto}{Cernusco s/N, Milano, Italia}{}{Ho lavorato in forma volontaria e gratuita in qualit\'a di insegnante dei principi della fede cattolica per ragazzi di scuola media.}
\cvitem{}{\small \textit{Lingua di lavoro:} italiano.}

\newpage

%----------------------------------------------%
%      ISTRUZIONE E CORSI DI FORMAZIONE        %
%----------------------------------------------%

\section{Istruzione e corsi di formazione}

%\cventry{Present}{PhD student at the Institute for Building Materials}{}{}{}{}
%\cvitem{}{\emph{Department of Civil, Environmental and Geomatic Engineering, ETH Z\"urich}}
%\cvitem{}{Z\"urich, Switzerland}

%------------------------------------------------

\subsection{Educazione superiore}

\cventry{Nov. 2013}{Abilitazione italiana alla Professione d'Ingegnere, settore Ingegneria Industriale}{}{}{}{}
\cvitem{}{\emph{Politecnico di Milano} - Milano, Italia}
\cvitem{}{\small \textit{Lingua di lavoro:} italiano.}

% Arguments not required can be left empty
\cventry{2010 - 2013}{Progetto di doppia laurea magistrale EAGLES \textit{(Engineers as Global Leaders for Energy Sustainability)}}{}{}{}{}
\cvitem{}{\emph{Politecnico di Milano - Drexel University - Universidad Polit\'ecnica de Madrid}}
\cvitem{}{Milano, Italia - Philadelphia, USA - Madrid, Spagna}
\cvitem{}{\small \textit{Lingua di lavoro:} italiano, inglese, spagnolo.}

\cventry{Ott. 2013}{Laurea Magistrale in Ingegneria Spaziale}{110/110}{}{}{}
\cvitem{}{\emph{Politecnico di Milano} - Milano, Italy}
\cvitem{}{\small \textit{Tesi di Laurea:} \href{https://www.politesi.polimi.it/handle/10589/82983}{Experimental, Analytical and Numerical Investigation of Loading Rate Effects on Mode I, Mode II and Mixed Mode I-II Delamination in Advanced CFRP.} \newline \href{https://www.politesi.polimi.it/handle/10589/82983}{(Studio sperimentale, analitico e numerico degli effetti della velocit\'a di carico sulla delaminazione in modo I, modo II e modo misto I-II in compositi avanzati).}}
\cvitem{}{\small \textit{Lingua di lavoro:} italiano, inglese, spagnolo.}
%\cvitem{}{\small \textit{Tesi di Laurea:} \href{https://www.politesi.polimi.it/handle/10589/82983}{Experimental, Analytical and Numerical Investigation of Loading Rate Effects on Mode I, Mode II and Mixed Mode I-II Delamination in Advanced CFRP.}}

%\cvitem{}{\small (Studio sperimentale, analitico e numerico degli effetti della velocit\'a di carico sulla delaminazione in modo I, modo II e modo misto I-II in compositi avanzati).}

\cventry{Giu. 2012}{Laurea Magistrale in Ingegneria Meccanica}{4/4}{}{}{}
\cvitem{}{\emph{Drexel University} - Philadelphia, USA}
\cvitem{}{\small \textit{Lingua di lavoro:} inglese.}

\cventry{2007 - 2010}{Laurea in Ingegneria Aerospaziale}{110/110}{}{}{}
\cvitem{}{\emph{Politecnico di Milano} - Milano, Italia}
\cvitem{}{\small \textit{Lingua di lavoro:} italiano.}

\cventry{2002 - 2007}{Diploma di Maturit\`a Scientifica}{100/100 con Lode}{}{}{}
\cvitem{}{\emph{Liceo Scientifico Istituto Sacro Cuore} - Milano, Italia}
\cvitem{}{\small \textit{Lingua di lavoro:} italiano.}

\cventry{1997 - 2007}{Studi di Violoncello}{}{}{}{}
\cvitem{}{\emph{Scuola Civica di Musica} - Cernusco s/N, Milano, Italia}
\cvitem{}{\small \textit{Lingua di lavoro:} italiano.}

\newpage

%------------------------------------------------

\subsection{Corsi di formazione e seminari}

\cventry{Nov. - Dic.}{Responsabilit\'a sociale d'impresa (CSR) e Creazione di Valore}{}{}{}{}
\cvitem{2015}{Organizzato da \emph{Audencia Nantes Business School}}
\cvitem{}{MOOC a distanza e Certificato di Completamento Verificato}
\cvitem{}{{\small Dopo una definizione introduttiva del concetto di Responsabilit\'a sociale d'impresa (Corporate Social Responsability, abbreviato CSR, nella letteratura anglosassone), il corso si \'e incentrato sull'approfondimento della relazione che intercorre tra quest'ultima e la capacit\'a di creare valore, elemento fondante dell'impresa. Problemi ed opportunit\'a per una realizzazione efficace di strategie orientate al CSR nell'ambito della gestione aziendale sono state presentate e discusse.}}
\cvitem{}{\small \textit{Lingua di lavoro:} inglese.}

\cventry{Lug. 2015}{Scuola Estiva 2015 sull'Utilizzo efficace delle Tecnologie di Calcolo ad elevate prestazioni}{}{}{}{}
\cvitem{}{Organizzato dal \emph{Centro Svizzero di Calcolo Scientifico} e \emph{Universit\'a della Svizzera \mbox{Italiana}}}
\cvitem{}{Ospitato da Hotel Serpiano, Ticino, Svizzera}
\cvitem{}{{\small Le attivit\'a della scuola erano incentrate sull'utilizzo efficace dei sistemi di calcolo HPC (High Performance Computing) e proponeva lezioni teoriche accompagnate da esercitazioni applicative. Fra gli argomenti principali: MPI, OpenMP, CUDA, OpenACC, I/O in parallelo, utilizzo di librerie scientifiche, visualizzazione di dati e tecniche di ottimizzazione.}}
\cvitem{}{\small \textit{Lingua di lavoro:} inglese.}

\cventry{Mag. 2015}{Stima degli Immobili e Ruolo dell'Esperto Stimatore}{}{}{}{}
\cvitem{}{Organizzato da \emph{Altalex Formazione}, Milano, Italia}
\cvitem{}{Riconosciuto dall' \emph{Ordine degli Ingegneri di Milano}}
\cvitem{}{{\small Seminario formativo sulle procedure di stima degli immobili e sul ruolo del tecnico nel corso della procedura di valutazione.}}
\cvitem{}{\small \textit{Lingua di lavoro:} italiano.}

\cventry{Mar. - Giu.}{Storia, Presente e Futuro dell'Unione Europea (Scuola di Cittadinanza Europea)}{}{}{}{}
\cvitem{2013}{Organizzato da \emph{CEPADE - Universidad Polit\'ecnica de Madrid}, Madrid, Spagna}
\cvitem{}{{\small Per celebrare l'\textit{Anno Europeo dei Cittadini 2013}, il corso ha proposto un'analisi in profondit\`a della storia e dello sviluppo politico, dell'attuale quadro istituzionale e delle prospettive future dell'Unione Europea.}}
\cvitem{}{\small \textit{Lingua di lavoro:} spagnolo.}

\cventry{Lug. 2012}{Scuola Estiva \textit{Wolfram Science 2012}}{}{}{}{}
\cvitem{}{Organizzato da \emph{Wolfram Research}}
\cvitem{}{Ospitato da Curry College, Milton, MA, USA}
\cvitem{}{{\small Le attivit\`a di apprendimento si sono concentrate sull'acquisizione di una conoscenza profonda del programma Mathematica e del suo relativo linguaggio di programmazione.}}
\cvitem{}{\small \textit{Progetto:} \href{https://www.wolframscience.com/summerschool/2012/alumni/stasio.html}{Modeling Complex Patterns of Crack Propagation: Branching and Merging Mechanisms.}\newline\href{https://www.wolframscience.com/summerschool/2012/alumni/stasio.html}{(Modellazione di configurazioni complesse di propagazione di fratture: meccanismi di separazione e di fusione.)}}
\cvitem{}{\small \textit{Lingua di lavoro:} inglese.}

\cventry{Ago. 2011}{Scuola Estiva \textit{Shanghai Summer School}}{}{}{}{}
\cvitem{}{Organizzato da \emph{China Engineering Education Excellence Alliance}}
\cvitem{}{Ospitato da Tongji University, Shanghai, Cina -- Northwestern
Polytechnical University, Xi'an, Cina -- Chongqing University, Chongqing, Cina}
\cvitem{}{{\small I principali temi trattati sono stati lo sviluppo urbano, la comunicazione internazionale, lo sviluppo sostenibile di aree urbane e rurali, lingua e cultura cinesi.}}
\cvitem{}{\small \textit{Lingua di lavoro:} inglese, cinese.}

\cventry{Marzo 2010}{Programma ATHENS {\small \textit{(Advanced Technology Higher Education Network/Socrates)}}}{}{}{}{}
\cvitem{}{Organizzato ed ospitato da \emph{Delft University of Technology}, Delft, Olanda}
\cvitem{}{{\small Partecipazione al corso intensivo \textit{Introduzione agli Elementi Finiti}, dedicato ai fondamenti e alla programmazione a livello introduttivo di metodi FEM.}}
\cvitem{}{\small \textit{Lingua di lavoro:} inglese.}

%------------------------------------------------

\subsection{Educazione professionale}

\cventry{Lug. - Dic.}{Tecnico di programmazione macchine a controllo numerico}{}{}{}{}
\cvitem{2009}{Held and hosted by \emph{Centro di Formazione Professionale Salesiano Don Bosco}, Milano, Italia}
\cvitem{}{Corso presenziale e Attestato di Competenza}
\cvitem{}{{\small Lezioni teoriche e tirocinio pratico su tornitura, fresatura, saldatura, programmazione ed uso di macchine a controllo numerico.}}
\cvitem{}{\small \textit{Lingua di lavoro:} italiano.}



%----------------------------------------------%
%          AFFILIAZIONI PROFESSIONALI          %
%----------------------------------------------%

\section{Affiliazioni professionali}

\cvitem{Dal 2014}{\textit{Ordine degli Ingegneri di Milano}}
\cvitem{Dal 2014}{\textit{American Society of Mechanical Engineers (ASME)}}
\cvitem{Dal 2013}{\textit{Society of Industrial and Applied Mathematics (SIAM)}}
\cvitem{Dal 2014}{\textit{Institute of Electrical and Electronics Engineers (IEEE)}}
%\cvitem{Dal 2014}{\textit{Swiss Physical Society (SPS)}}

%----------------------------------------------%
%           PREMI ED ONORIFICENZE              %
%----------------------------------------------%

\section{Premi ed onorificenze}

\cvitem{2015 - Presente}{\textit{Borsa di ricerca Erasmus Mundus finanziata dalla Commissione Europea}, per la partecipazione al programma di dottorato DocMASE.}
\cvitem{Feb. 2015}{\textit{BCC (Banca di Credito Cooperativo) di Cernusco s/N}, premio per eccellenza negli studi di laurea magistrale.}
\cvitem{Ott. 2013}{\textit{Onorificenza PEGASUS}, per speciali meriti nel campo della cooperazione europea.}
\cvitem{2012 - 2013}{\textit{Programma Erasmus}, borsa di studio per un periodo di ricerca presso IMDEA Materials.}
\cvitem{2011 - 2012}{\textit{Borsa di studio del Programma EU-US Atlantis finanziata dal Dipartimento dell'Educazione statunitense e dalla Commissione Europea}, copertura totale della retta universitaria e borsa di studio per gli studi di laurea magistrale presso Drexel University.}
\cvitem{Marzo 2012}{\textit{BCC (Banca di Credito Cooperativo) di Cernusco s/N}, premio per eccellenza negli studi di laurea.}
\cvitem{2007 - 2010}{\textit{Politecnico di Milano}, copertura parziale della retta universitaria per gli studi di laurea.}
\cvitem{Marzo 2008}{\textit{BCC (Banca di Credito Cooperativo) di Cernusco s/N}, premio per eccellenza negli studi di scuola secondaria superiore.}
\cvitem{Set. 2007}{\textit{Governo italiano}, premio per eccellenza negli studi di scuola secondaria superiore.}

%----------------------------------------------%
%          COMPETENZE LINGUISTICHE             %
%----------------------------------------------%

\section{Competenze linguistiche}

\cvitemwithcomment{Italiano}{Lingua madre}{}
\cvitemwithcomment{Inglese}{Ottimo}{}
\cventry{}{\small IELTS Certificate}{\small 8.5/9.0}{\small \href{http://en.wikipedia.org/wiki/Common_European_Framework_of_Reference_for_Languages}{Livello CEF} C2}{\small Mag. 2012}{}
\cventry{}{\small TOEIC Certificate}{\small 945/990}{\small \href{http://en.wikipedia.org/wiki/Common_European_Framework_of_Reference_for_Languages}{Livello CEF} C1}{\small Set. 2010}{}
\cventry{}{\small FCE of Cambridge University}{B}{\small \href{http://en.wikipedia.org/wiki/Common_European_Framework_of_Reference_for_Languages}{Livello CEF} B2}{\small 2006}{}
\cvitem{}{{\small Eccellenti capacit\`a di lettura ed ascolto, eccellente abilit\`a nella comunicazione scritta ed orale.}}

\cvitemwithcomment{Spagnolo}{Intermedio}{}
\cvitem{}{{\small Ottime capacit\`a di lettura ed ascolto, buona abilit\`a nella comunicazione scritta ed orale.}}
\cvitemwithcomment{Tedesco}{Discreto}{}
\cventry{}{\small Fit in Deutsch 1 - Goethe Institut}{\small Sehr gut - 53.50/60}{\small \href{http://en.wikipedia.org/wiki/Common_European_Framework_of_Reference_for_Languages}{Livello CEF} A1}{\small 2002}{}
\cvitem{}{{\small Buone capacit\`a di lettura ed ascolto, discreta abilit\`a nella comunicazione scritta ed orale.}}
%\cvitem{}{{\small Nel 2002 ho ottenuto il Fit in Deutsch 1 del Goethe Institut, corrispondente al livello A1 del Quadro Comune Europeo.}}
\cvitemwithcomment{Francese}{Discreto}{}
\cvitem{}{{\small Buone capacit\'a di lettura ed ascolto, discreta abilit\`a nella comunicazione scritta ed orale.}}
\cvitemwithcomment{Russo}{Elementare}{}
\cvitem{}{{\small Discrete capacit\`a di lettura ed ascolto, discreta abilit\`a nella comunicazione scritta ed orale.}}
\cvitem{}{{\small Da ottobre 2010 a maggio 2011 ho frequentato un corso di russo presso l'Associazione Italia-Russia - Milano, Italia, riconosciuta ufficialmente dall'ambasciata russa.}}
\cvitemwithcomment{Cinese}{Basico}{}
\cvitem{}{{\small In agosto 2011, ho studiato le basi della lingua cinese durante la Shanghai Summer School.}}


%----------------------------------------------%
%            COMPETENZE INFORMATICHE           %
%----------------------------------------------%

\section{Competenze informatiche}

\cvitem{}{\textit{Sistemi operativi:} Windows, Linux, Mac OS.}
\cvitem{}{\textit{Linguaggi di programmazione:} C, C++, Python, Perl, Java, Fortran, HTML, bash.}
\cvitem{}{\textit{Calcolo HPC:} OpenMP, MPI, CUDA, OpenACC.}
\cvitem{}{\textit{Calcolo scientifico:} Matlab, Mathematica, Stata, Octave, Maple.}
\cvitem{}{\textit{Modellazione e simulazione fisica:} Abaqus, Comsol Multiphysics, Thermal Desktop.}
\cvitem{}{\textit{Visualizzazione di dati:} ParaView, VisIt.}
\cvitem{}{\textit{Programmi CAD/CAM:} Autocad, Solid Edge, codici G-M-T-S-F.}
\cvitem{}{\textit{Preparazione di documenti:} LaTex, JabRef, EndNote.}
\cvitem{}{\textit{Programmi d'uso generale:} pacchetto Office (Word, Excel, Powerpoint) ed equivalenti, navigazione Internet (Google Chrome, Firefox, Internet Explorer), creazione e modifica PDF, creazione e modifica d'immagini.}


%----------------------------------------------%
%            INTERESSI PERSONALI               %
%----------------------------------------------%

\section{Interessi personali}

\cvitem{}{Sto attualmente studiando chitarra nel mio tempo libero. Sono interessato alla politica ed alla finanza. Adoro leggere, specialmente classici; ascoltare musica sia classica che contemporanea; sono appassionato di teatro e cinema. Sono amante della pittura e della scultura, e dell'arte in generale. Mi piace praticare sport, gioco spesso a calcio e pratico con regolarit\`a attivit\`a fisica.}
\cvitem{}{Partecipo con regolarit\`a ad iniziative di volontariato. A Philadelphia, ho preso parte alle attivit\`a dell'associazione Philabundance, dedicate alla raccolta di cibo per persone in difficolt\`a. Durante il mio soggiorno in Madrid, ho partecipato ad opere di carit\`a per i senzatetto della citt\`a. A Zurigo, ho preso parte ad attivit\`a gratuite di cura ed intrattenimento di bambini.}

\section{Per sapere di pi\'u su di me}
\cvitem{Segui}{il codice QR}
\cvitem{}{\includegraphics[width=0.2\textwidth]{pictures/qrcode.eps}}
\cvitem{o clicca}{sul link per sapere di pi\'u \href{http://www.lucadistasioengineering.com/}{\textit{su di me.}}}

%%%%%%%%%%%%%%%%%%%%%%%%%%%%%%%%%%%%%%%%%%%%%%%%%%%%%%%%%%%%%%%%%%%%%%%%%%%%%%%%%%%%%%%%%%%%%%%%%%%%%%
%%%%%%%%%%%%%%%%%%%%%%%%%%%%%%%%%%%%%%%%%%%%%%%%%%%%%%%%%%%%%%%%%%%%%%%%%%%%%%%%%%%%%%%%%%%%%%%%%%%%%%
%%--------------------------------------------------------------------------------------------------%%
%%---------------------------------- FINE DEL DOCUMENTO --------------------------------------------%%
%%--------------------------------------------------------------------------------------------------%%
%%%%%%%%%%%%%%%%%%%%%%%%%%%%%%%%%%%%%%%%%%%%%%%%%%%%%%%%%%%%%%%%%%%%%%%%%%%%%%%%%%%%%%%%%%%%%%%%%%%%%%
%%%%%%%%%%%%%%%%%%%%%%%%%%%%%%%%%%%%%%%%%%%%%%%%%%%%%%%%%%%%%%%%%%%%%%%%%%%%%%%%%%%%%%%%%%%%%%%%%%%%%%

\end{document}